%%%%%%%%%%%%%%%%%%%%%%%%%%%%%%%%%%%%%%%%%%%%%%%%%%%%%%%%%%%%%%%%%%%%%%%%%%%%%%%%%%%%%%%
%%%
%%% Please do not remove this header.
%%%
%%% This is official version of the template that should be used to prepare 
%%% a manuscript to be submitted to the HNPS Advances in Nuclear Physics
%%% (ISBN print: 2654-007X, online: 2654-0088; abbrv. HNPS Adv. Nucl. Phys.)
%%% The journal is the official publication of the Hellenic Nuclear Physics Society
%%% established in 1990 and is indexed in Scopus
%%% Information about the journal (incl. submission links) can be found here: 
%%% https://eproceedings.epublishing.ekt.gr/index.php/hnps/index
%%%
%%% Template created by Theo J. Mertzimekis (tmertzi@phys.uoa.gr)
%%% v.1 2025-06-06
%%% 
%%%%%%%%%%%%%%%%%%%%%%%%%%%%%%%%%%%%%%%%%%%%%%%%%%%%%%%%%%%%%%%%%%%%%%%%%%%%%%%%%%%%%%%
\documentclass[
  manuscript=article,  %% article
%  layout=preprint,  %% preprint (for submission) or publish (for publisher only and after acceptance)
  layout=publish,  %% preprint (for submission) or publish (for publisher only and after acceptance)
  year=20XX,
  volume=XX,
%  IssueNumber=x
]{extras/hnpsanp}


\doi{~10.12681/hnpsanp.XXXX}

% only used by journal admin (changed to reflect the first page of the article)
\StartPage{1}


\received {1 April 20xx}
\accepted {10 May 20xx}
\published{20 May 20xx}



% --- blew is the area for authors ---

% remove the following two packages, and delete all \blindtext commands
\usepackage[english]{babel} 
\usepackage{blindtext}
\usepackage[right]{lineno}
\usepackage{caption}
\usepackage{subcaption}
\usepackage{isotope}

% specify the .bib file for references
\addbibresource{HNPS-refs.bib} 


% Make sure your article tile is within 12 words
\title{Using a concise title for your HNPS Adv. Nucl. Phys. article. Try to stay within 2 lines.}

\author{F. Author}
\affiliation{Institution-1, City, Country}
\email{correspondence@email.domain}

\author{S.B. Author}
\affiliation{Institution-2, City, Country}

\author{T.-H. Author}
\alsoaffiliation{Institution-1, City, Country}
\affiliation{Institution-3, City, Country}


% maximum five keywords
\keywords{keyword 1; keyword 2; keyword-3 (include max 5)} 



\begin{document}

\linenumbers

\begin{abstract}
  An abstract summarizes in one paragraph with 300 words or less, the major aspects of the entire paper. They often include: 1) the overall purpose of the study and the research problem you investigated; 2) the basic design of you research approach; 3) major findings as a result of your analysis; and, 4) a brief summary of your interpretations and conclusions. 
\end{abstract}


\section{Introduction}

\blindtext 
open data in science~\cite{Mavrommatis_Clark_1990}.


\blindtext [2]


\section{Materials and Methods}

\subsection{Methods part 1}

\blindtext The end result is in Fig.~\ref{fig:logo}.

\begin{figure}[ht!]
  \centering
  \includegraphics[width=0.45\textwidth]{example-image}
  \caption{An Example Figure}
  \label{fig:logo}
\end{figure}

\blindtext\footnote{This is how a footnote works. Remove this line in the source code if you do not need it.}

\subsection{Methods part 2}

\subsubsection{Methods part 2.1}
\blindtext
This is an example of a figure with multiple subfigures. The user can define the relative widths and subcaptions.
\begin{figure}
     \centering
     \begin{subfigure}[b]{0.3\textwidth}
         \centering
         \includegraphics[width=\textwidth]{extras/logo_hnpsanp_header.png}
         \caption{Subfigure 1}
         \label{fig:subfig1}
     \end{subfigure}
     \hfill
     \begin{subfigure}[b]{0.3\textwidth}
         \centering
         \includegraphics[width=\textwidth]{extras/logo_hnpsanp_header.png}
         \caption{Subfigure 2}
         \label{fig:subfig2}
     \end{subfigure}
     \hfill
     \begin{subfigure}[b]{0.3\textwidth}
         \centering
         \includegraphics[width=\textwidth]{extras/logo_hnpsanp_header.png}
         \caption{Subfigure 3}
         \label{fig:subfig3}
     \end{subfigure}
        \caption{Three simple graphs}
        \label{fig:main_figure_label}
\end{figure}

Below are some examples on how you should write equations. Here is typical example of an equation~\eqref{eq:cauchy_momentum}
%
\begin{equation}
\rho\frac{\mathrm{D} \mathbf{u}}{\mathrm{D} t} = - \nabla p + \nabla \cdot \boldsymbol \tau + \rho\,\mathbf{g}
\label{eq:cauchy_momentum}
\end{equation}
%
Here is an example of a single equation without numbering:
\[
\cos^2x+\sin^2x=1
\label{eq:trigon}
\]
%
Here is an example of multiple equations, aligned at the `=` sign and numbering only at the last equation of the bunch.
%
\begin{align}
    3x^2-4y+2 &=e^{y^2} \nonumber\\
    3x^2 &=4y-2+e^{y^2}\nonumber\\
    x &=\pm\sqrt{\frac{1}{3}\left(2+4y+e^{y^2}\right)}
    \label{eq:multi}
\end{align}

\subsubsection{Methods part2.2}

\blindtext Table \ref{tb:example_table} shows an example.

\begin{table}[H]
  \centering
  \small
  \caption{Example table}
  \label{tb:example_table}
  \begin{tabular}{lll}
  \toprule
  \textbf{Parameter} & \textbf{Notation} & \textbf{Remarks} \\
  \midrule
  name & - & engine common identifier \\
  manufacture & - & name of the manufacture  \\
  bpr & $\lambda$ & bypass ratio \\
  pr & - & pressure ratio \\
  thrust & $T_0$ & maximum static thrust\\
  \bottomrule
  \end{tabular}
\end{table}

\blindtext


\section{Results and Discussion}

\paragraph{Paragraph title} This is the paragraph with title if you want to use such function in the paper. \blindtext


\section{Conclusion}

\blindtext


\section*{How to prepare your Reference list}

The Reference list is prepared using the bib\TeX\ format. A HNPS-refs.bib file is provided in this template as a basic example.
Modern literature management software such as Zotero, JabRef, Mendeley etc provide .bib files from their Export function. Please
refer to \href{https://www.overleaf.com/learn/latex/Bibliography_management_in_LaTeX}{this guide} on how to prepare, cite and
format your lists of citations. References are shown in a numeric style, here is an example~\cite{python}.

%\section{Supplementary tables}
%\blindtext


\section*{Acknowledgements}
Include your acknowledgement in this section.


\section*{Funding statement}
When applicable, please specify the funding information for this research, else remove the section completely



% the bibliography is included in the HNPS-refs.bib file
\scriptsize
\printbibliography
\normalsize

\appendix

\section{This is an appendix, if you need it}

This is an appendix section to provide supplementary material, if you wish. An appendix can have multiple chapters and sections. Here it includes some text and a version of the HNPS logo, shown in Fig.~\ref{fig:appx}.

\begin{figure}[H]
\centering
\includegraphics[width=0.5\textwidth]{extras/logo_hnpsanp_header.png}
\caption{The HNPS ANP logo}
\label{fig:appx}
\end{figure}


\end{document}